%----------------------------------------------------------------------------
% Course Syllabus for Chemistry 352 (c) 2015 by Dale J. Brugh

% Course Syllabus for Chemistry 352 is released under a
% Creative Commons Attribution-ShareAlike 4.0 International License.

% See http://creativecommons.org/licenses/by-sa/4.0/ for
% a description of your rights under this license. 
%----------------------------------------------------------------------------
\documentclass[letterpaper,oneside,onecolumn,11pt,article]{memoir}
%
% --- LOAD PACKAGES ---
%
\usepackage[T1]{fontenc}            % use T1 font encoding
\usepackage{textcomp}
\usepackage{courier}                % set courier as typewriter font
\usepackage{times}                  % set times as text font
\usepackage[scaled=0.92]{helvet}    % set Helvetica as the sans-serif font
\usepackage{mtpro2}
\usepackage{setspace}
\usepackage{amsmath}
\usepackage{graphicx,color}
\usepackage{wallpaper}
\usepackage{textcomp}
\usepackage{relsize,fancyvrb}
\usepackage{verbatim}
\usepackage{caption}
\usepackage{paralist}
\usepackage{boxedminipage}
\usepackage{enumerate}
\usepackage[bookmarks=true]{hyperref}
%
% --- HYPER SETUP ---
%
\hypersetup{
    unicode=false,          % non-Latin characters in Acrobat’s bookmarks
    pdftoolbar=true,        % show Acrobat’s toolbar?
    pdfmenubar=true,        % show Acrobat’s menu?
    pdffitwindow=true,      % page fit to window when opened
    pdftitle={Syllabus: Chemistry 352 / Spring 2015}, 
    pdfauthor={Dale J. Brugh},     % author
    pdfsubject={Physical Chemistry},   % subject of the document
    pdfnewwindow=true,      % links in new window
    pdfkeywords={classes, ch352s15}, % list of keywords
    colorlinks=true,       % false: boxed links; true: colored links
    linkcolor=black,          % color of internal links
    citecolor=green,        % color of links to bibliography
    filecolor=magenta,      % color of file links
    urlcolor=black           % color of external links
}
\definecolor{nicered}{rgb}{.647,.129,.149}
\definecolor{mutedgrey}{rgb}{0.4,0.4,0.4}
\definecolor{shadecolor}{cmyk}{0,0,0.25,0.07}
\definecolor{MyDarkBlue}{rgb}{0,0.08,0.45}
\definecolor{MarginRed}{rgb}{0.8,0.0,0.0}
\definecolor{MarginBlue}{rgb}{0.2,0.0,1.0}
\definecolor{MarginGrey}{rgb}{0.4,0.4,0.4}
%\renewcommand{\chapnumfont}{\bfseries\Huge\sffamily}
%\renewcommand{\chaptitlefont}{\bfseries\Large\sffamily}
\setsecheadstyle{\bfseries\Large\sffamily\raggedright}
\setsubsecheadstyle{\bfseries\large\sffamily\raggedright}
\setsubsubsecheadstyle{\bfseries\normalsize\sffamily\raggedright}
\renewcommand \thesection{\bfseries\arabic{section}}
\makeatletter 
\newcommand\addRevisionData{%
\begin{picture}(0,0)% 
    \put(-110,-5){%
        \tiny% 
        {%
        {Published \today \enspace \copyright~Dale J. Brugh 
        }
}% 
}%
\end{picture}%
}
\flushbottom
\setstocksize{11in}{8.5in}
%\setlength{\parskip}{5pt}
\settrims{0pt}{0pt}
%\settrimmedsize{11in}{210mm}{*}
%\setlength{\trimtop}{0pt}
%\setlength{\trimedge}{\stockwidth}
%\addtolength{\trimedge}{-\paperwidth}
\settypeblocksize{8.5in}{5.0in}{*}
\setulmargins{1.25in}{*}{*}
\setlrmargins{1.25in}{*}{*}
\setmarginnotes{5mm}{4.0cm}{\onelineskip}
\setheadfoot{\onelineskip}{4\onelineskip}
\setheaderspaces{*}{\onelineskip}{*}
\checkandfixthelayout
%\setlength \fboxsep{0.1in}
\setlength \headwidth{\textwidth+\marginparwidth+\marginparsep}
%
\makepagestyle{courseinformation}
\makerunningwidth{courseinformation}{\headwidth}
\makeheadrule{courseinformation}{\headwidth}{\normalrulethickness}
\makeheadposition{courseinformation}{flushright}{flushleft}{flushleft}{flushleft}
\makeoddhead{courseinformation}%
    {\sffamily Course Syllabus: Chemistry 351 / Spring 2015}{}{\sffamily\thepage}

    \makeevenfoot{courseinformation}{}{}{}
    \makeoddfoot{courseinformation}{}{}{}
\makepagestyle{courseinformationtitle}
\makerunningwidth{courseinformationtitle}{\headwidth}
\makeheadposition{courseinformationtitle}{flushright}{flushleft}{flushleft}{flushleft}
    \makeevenfoot{courseinformationtitle}{}{}{}
    \makeoddfoot{courseinformationtitle}{}{}{}
\pagestyle{courseinformation}
\captionsetup{labelsep=colon,aboveskip=0.25cm,justification=RaggedRight,singlelinecheck=false,labelfont={bf,sf}}
%
% --- MARGIN FIGURE COMMAND ---
%
\newcommand{\marginfigures}[4]{
\marginpar{\centering
\includegraphics[width=#1]{#2}
\captionsetup{labelsep=newline,aboveskip=-0.5cm,justification=RaggedRight,singlelinecheck=false,labelfont={bf,sf}}
\captionsetup[figure]{position=bottom}
\captionof{figure}{#3}
\label{#4}}%
}%
%
% --- MARGIN NOTE COMMAND ---
%
\newcommand{\marginnote}[2]
{%
\marginpar{\raggedright\vspace{#1}\begin{Spacing}{0.65}\sffamily{{\tiny$\blacktriangleright$~\scriptsize#2}}\end{Spacing}} %
}
%
% --- SET UP TITLE ---
%
\setlength{\droptitle}{0.0in}
\backmatter
\pretitle{\noindent\huge\sffamily Course Syllabus \LARGE\par\noindent} 
\posttitle{\par\vskip 2.0em}
\preauthor{}
\postauthor{\par}
\predate{}
\postdate{\noindent\rule{\linewidth}{0.3pt}}
\title{Chemistry 351 / Spring 2015}
\date{}
\author{}
%
% --- BEGIN DOCUMENT ---
%
\begin{document}
\setsecnumdepth{subsubsection}
\maketitle
%\setsecnumdepth{subsection}
\thispagestyle{courseinformationtitle}
%
% --- INSTRUCTOR ---
%
\section{Instructor}
\begin{tabular}{rl|rl}
Name: & Dr. Dale J. Brugh & Office Phone: & 740-368-3530 \\
Email: & \href{mailto:djbrugh@owu.edu}{djbrugh@owu.edu} & Cell Phone: & 614-746-2397 \\
Office: & SCSC 262 & &  \\
\end{tabular}
%
% --- MEETINGS ---
%
\section{Meetings}
\begin{tabular}{crcrl}
M & 1:10 p.m. & to & 4:00 p.m. & SCSC 355 \\
\end{tabular}

\section{Course Website}
The course web site is at http://dephlo.net/pchemlab. The course website is an important extension of this syllabus and should be read carefully. 

\section{Laboratory Materials}
The following items will be used in this laboratory. 
\begin{enumerate}[1.]
\item \textit{Exploring Chemistry with Electronic Structure Methods} by James B. Foresman and \AE leen Frisch, Second Edition, Gaussian, Inc. (1996). ISBN-10: 0963676938, ISBN-13: 978-0963676931.
\item \emph{Physical Chemistry} by Thomas Engel and Phillip Reid, Third Edition, Prentice Hall (2013).  ISBN-13: 978-0321812001
\item \emph{Quanta, Matter, and Change} by Peter Atkins, Julio de Paula, and Ronald Friedman, First Edition, W.H. Freeman and Company (2009). ISBN-10: 07167-6117-3, ISBN-13: 978-0-7167-6117-4.
\end{enumerate}

\section{Laboratory Prerequisites}
To take this laboratory course you must have passed one semester of physical chemistry that covers quantum mechanics. The easiest way to meet this prerequisite is to have credit for CHEM350. 

\section{Laboratory Content}
Chemistry 352 is an introduction to molecular modeling and electronic structure methods in chemistry using Gaussian 09. The emphasis is on answering chemical questions using available tools. The topics to be covered are listed in Table~\ref{tab:topics}. These may not be covered in this order, and not all topics will be covered with equal depth. 

\begin{table}[h]
\caption{\sffamily Topics covered in Chemistry 352.}
\label{tab:topics}
\begin{tabular}{l|l} \toprule
Using Gaussian & Basis Sets \\
Specifying Molecular Structure & Selecting Methods \\
Energy Calculations & Chemical Reactivity \\
Geometry Optimization & Excited States \\
Frequency Calculations & Solutions \\ 
\bottomrule
\end{tabular}
\end{table}
%
% --- LABORATORY GOALS ---
%
\section{Laboratory Goals}
This laboratory has several goals. Each action I take is designed to move us closer to achieving one or all of these goals. My goals in this laboratory are
\begin{enumerate}[1.]
\item To provide you with an overview of molecular modeling and electronic structure techniques for computing properties of molecules,
\item To give you practice carrying out molecular modeling and electronic structure calculations on molecules using a research-grade tool,
\item To make you better able to recognize, understand, and interpret molecular modeling and electronic structure results presented in the literature,
\item To make you better able to formulate chemical questions that can be answered with molecular modeling and electronic structure methods, and
\item To provide you with the tools to continue your education in chemistry in the very best graduate programs on planet Earth.
\end{enumerate}
%
% --- LABORATORY EXPECTATIONS ---
%
\section{Laboratory Expectations}
You are expected to read the assigned readings and to come to each laboratory meeting prepared to participate. You are expected to display a high level of independence and a strong willingness to solve problems through your own troubleshooting. In short, you are expected to be professional. It is assumed that you are passionate about chemistry.
%
% --- WEEKLY ROUTINE ---
%
\section{Weekly Routine}
During \marginnote{-0.1in}{All projects are completed as part of an assigned team.} laboratory meetings you will work on projects intended to illustrate particular concepts in computational chemistry and to challenge your problem solving abilities. Projects will span one or two weeks. A pre-lab reading will typically be assigned for each project. A report will be due upon completion of each project. For the last one-third of the semester, laboratory meetings will be dedicated to work on a single independent project of your choosing. A detailed schedule can be found on the course website.
%
% --- THINGS I GRADE ---
%
\section{Things I Grade}
You and I will determine your progress in this laboratory using the scores derived from evaluating your project reports, final project, and final exam. This section provides details for each of these.
%
% --- Project Reports ---
%
\subsection{Project Reports}
For \marginnote{-0.1in}{Each team submits one report, and all members of the team receive the same score.} each laboratory project, you must complete a written report consisting of three sections: abstract, procedure, and results. You must also include references. Reports will be evaluated for clarity, presentation, and accuracy. Work presented in reports must be substantiated by your laboratory notebook. Details for preparing reports and their due dates and times can be found on the course website. Each report will be evaluated on a 200-point scale. Your lowest single laboratory report score is dropped before computing your laboratory score. 
%
% --- Final Project ---
%
\subsection{Final Project}
During \marginnote{-0.1in}{The final project may completed as part of a team or individually.} the last third of the semester, laboratory meetings will be dedicated to work on a self-designed final project. Your work on this final project must be presented in a formal written report and an oral presentation. The combination of your formal report, oral presentation, and supporting work will be evaluated on a 200 point scale with 100 points dedicated to the report and 100 points dedicated to the presentation. Work presented in the final report and presentation must be substantiated by your laboratory notebook. 
%
% --- Final Exam ---
%
\subsection{Final Exam}
A final exam will be given during one of the final exam time slots for MWF afternoon courses. The final exam will consist of questions about carrying out molecular modeling and electronic structure calculations. It will also consist of a practical in which you have to carry out computations. You can use your laboratory notebook during the exam; it is therefore in your best interest to keep a detailed record of your laboratory work. 
%
% --- LABORATORY NOTEBOOK ---
%
\section{Laboratory Notebook}
Each of you must maintain a laboratory notebook as a complete record of all work in and out of the laboratory. Guidelines for keeping a laboratory notebook can be found on the course web site. I will not evaluate your laboratory notebook, but all evaluated work must be supported by your laboratory notebook. Points may be lost on evaluated work if your claims are not supported by your laboratory notebook. 
%
% --- COURSE LETTER GRADE ---
%
\section{Course Letter Grade}
Your \marginnote{-0.1in}{Evaluation Weights \\ Project Reports: 30\% \\ Final Project: 40\% \\ Final Exam: 30\%} course score will be calculated as a weighted average of the scores you earn on evaluation of the project reports ($30\%$), final project ($40\%$), and final exam ($30\%$). 
%
% --- LETTER GRADE ---
%
\section{Letter Grade}
Letter grades are assigned at the end of the course according to the minimum course score requirements listed in Table~\ref{tab:lettergrades}. Course scores below $55\%$ are considered failing. Please see \href{http://dephlo.net/lettergrades}{dephlo.net/lettergrades} for more detail about how your course letter grade is determined. 
\begin{table}[h]
\caption{\sffamily Minimum course scores necessary for each letter grade.}
\label{tab:lettergrades}
\begin{tabular}{cl||cl} \toprule
\textbf{Minimum Score} & \textbf{Letter Grade} & \textbf{Minimum Score} & \textbf{Letter Grade} \\ \hline
97 & \hspace{0.3in}A$+$ & 72 & \hspace{0.3in}C$+$ \\
88 & \hspace{0.3in}A & 68 & \hspace{0.3in}C \\
85 & \hspace{0.3in}A$-$ & 65 & \hspace{0.3in}C$-$ \\
82 & \hspace{0.3in}B$+$ & 62 & \hspace{0.3in}D$+$ \\
78 & \hspace{0.3in}B & 58 & \hspace{0.3in}D \\
75 & \hspace{0.3in}B$-$ & 55 & \hspace{0.3in}D$-$ \\
\bottomrule
\end{tabular}
\end{table}
%
% --- ADDITIONAL INFORMATION ---
%
\section{Additional Information}
Please see the course website at \href{http://dephlo.net/pchemlab}{dephlo.net/pchemlab} for additional information such as suggestions for success, detailed course policies, problem set guidelines, course schedule, and solutions. 
%
% ---LICENSE AND SOURCE CODE ---
%
\section{License and Source Code}
\copyright\ 2015 by Dale J. Brugh. Course Syllabus for Chemistry 352 (2015) is made available under a Creative Commons Attribution-ShareAlike 4.0 International License (CC BY-SA 4.0). See \href{https://creativecommons.org/licenses/by-sa/4.0/}{https://creativecommons.org/licenses/by-sa/4.0/} for details about your rights under this license. The \LaTeX\ source code is available under the same license from Github at \href{https://github.com/djbrugh/chem352}{https://github.com/djbrugh/chem352}. \marginpar{\vspace{-0.85in}\hfill\noindent\href{https://creativecommons.org/licenses/by-sa/4.0/}{\includegraphics[width=1in]{figs/cc-by-sa.pdf}}}


\end{document}