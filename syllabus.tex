%----------------------------------------------------------------------------
% Chemistry 352 Syllabus
% Author: Dale J. Brugh
%----------------------------------------------------------------------------
\documentclass[letterpaper,oneside,onecolumn,11pt,article]{memoir}
\usepackage[T1]{fontenc}            % use T1 font encoding
\usepackage{textcomp}
\usepackage{courier}                % set courier as typewriter font
\usepackage{times}                  % set times as text font
\usepackage[scaled=0.92]{helvet}    % set Helvetica as the sans-serif font
\usepackage{mtpro2}

\usepackage{setspace}
\usepackage{amsmath}
\usepackage{graphicx,color}
\usepackage{wallpaper}
\usepackage{textcomp}
\usepackage{relsize,fancyvrb}
\usepackage{verbatim}
\usepackage{caption}

\usepackage{boxedminipage}

\usepackage{hyperref}
\usepackage[filehooks]{svn-multi} % For typesetting svn Id
\svnid{$Id: courseinfo.tex 911 2012-01-11 14:17:14Z djbrugh $}
\hypersetup{
    bookmarks=true,         % show bookmarks bar?
    unicode=false,          % non-Latin characters in Acrobat’s bookmarks
    pdftoolbar=true,        % show Acrobat’s toolbar?
    pdfmenubar=true,        % show Acrobat’s menu?
    pdffitwindow=true,      % page fit to window when opened
    pdftitle={Course Information: Chemistry 352 / Spring 2014}, 
    pdfauthor={Dale J. Brugh},     % author
    pdfsubject={Physical Chemistry},   % subject of the document
    pdfnewwindow=true,      % links in new window
    pdfkeywords={classes, ch352s14}, % list of keywords
    colorlinks=true,       % false: boxed links; true: colored links
    linkcolor=black,          % color of internal links
    citecolor=green,        % color of links to bibliography
    filecolor=magenta,      % color of file links
    urlcolor=cyan           % color of external links
}


\definecolor{nicered}{rgb}{.647,.129,.149}
\definecolor{mutedgrey}{rgb}{0.4,0.4,0.4}
\definecolor{shadecolor}{cmyk}{0,0,0.25,0.07}
\definecolor{MyDarkBlue}{rgb}{0,0.08,0.45}
\definecolor{MarginRed}{rgb}{0.8,0.0,0.0}
\definecolor{MarginBlue}{rgb}{0.2,0.0,1.0}
\definecolor{MarginGrey}{rgb}{0.4,0.4,0.4}

%\renewcommand{\chapnumfont}{\bfseries\Huge\sffamily}
%\renewcommand{\chaptitlefont}{\bfseries\Large\sffamily}

\setsecheadstyle{\bfseries\Large\sffamily\raggedright}
\setsubsecheadstyle{\bfseries\large\sffamily\raggedright}
\setsubsubsecheadstyle{\bfseries\normalsize\sffamily\raggedright}
\renewcommand \thesection{\bfseries\arabic{section}}

% \makeatletter 
% \newcommand\addRevisionData{%
% \begin{picture}(0,0)% 
%   \put(-125,-10){%
%       \tiny% 
%       \expandafter\@ifmtarg\expandafter{\svnfiledate}{}{%
%       {\copyright~Dale J. Brugh 
%       \svnfileyear-\svnfilemonth-\svnfileday
%       \enspace (revision \svnfilerev)}
% }% 
% }%
% \end{picture}%
% }

\flushbottom
\setstocksize{11in}{8.5in}
%\setlength{\parskip}{5pt}
\settrims{0pt}{0pt}
%\settrimmedsize{11in}{210mm}{*}
%\setlength{\trimtop}{0pt}
%\setlength{\trimedge}{\stockwidth}
%\addtolength{\trimedge}{-\paperwidth}
\settypeblocksize{8.5in}{5.0in}{*}
\setulmargins{1.25in}{*}{*}
\setlrmargins{1.25in}{*}{*}
\setmarginnotes{5mm}{4.0cm}{\onelineskip}
\setheadfoot{\onelineskip}{4\onelineskip}
\setheaderspaces{*}{\onelineskip}{*}
\checkandfixthelayout
%\setlength \fboxsep{0.1in}
\setlength \headwidth{\textwidth+\marginparwidth+\marginparsep}
%
\makepagestyle{courseinformation}
\makerunningwidth{courseinformation}{\headwidth}

\makeheadrule{courseinformation}{\headwidth}{\normalrulethickness}
\makeheadposition{courseinformation}{flushright}{flushleft}{flushleft}{flushleft}
\makeoddhead{courseinformation}%
    {\sffamily Course Information: Chemistry 352 / Spring 2014}{}{\sffamily\thepage}

    \makeevenfoot{courseinformation}{}{}{}
    \makeoddfoot{courseinformation}{}{}{}

\makepagestyle{courseinformationtitle}
\makerunningwidth{courseinformationtitle}{\headwidth}
\makeheadposition{courseinformationtitle}{flushright}{flushleft}{flushleft}{flushleft}
    \makeevenfoot{courseinformationtitle}{}{}{}
    \makeoddfoot{courseinformationtitle}{}{}{}

\pagestyle{courseinformation}

\captionsetup{labelsep=colon,aboveskip=0.25cm,justification=RaggedRight,singlelinecheck=false,labelfont={bf,sf}}

%===========================================================================
% Margin Figure Command
%===========================================================================
\newcommand{\marginfigures}[4]{
\marginpar{\centering
\includegraphics[width=#1]{#2}
\captionsetup{labelsep=newline,aboveskip=-0.5cm,justification=RaggedRight,singlelinecheck=false,labelfont={bf,sf}}
\captionsetup[figure]{position=bottom}
\captionof{figure}{#3}
\label{#4}}%
}%

%==========================================================================
% Margin Note Command
%==========================================================================

\newcommand{\marginnote}[2]
{%
\marginpar{\raggedright\vspace{#1}\begin{Spacing}{0.65}\sffamily{{\tiny$\blacktriangleright$~\scriptsize#2}}\end{Spacing}} %
}

%===========================================================================
% Set Up The Title
%===========================================================================
\setlength{\droptitle}{0.0in}
\backmatter
\pretitle{\noindent\huge\sffamily Course Information \LARGE\par\noindent} 
\posttitle{\par\vskip 2.0em}
\preauthor{}
\postauthor{\par}
\predate{}
\postdate{\noindent\rule{\linewidth}{0.3pt}}

\title{Chemistry 352 / Spring 2014}
\date{}
\author{}

%============================================================================
% Begin Document
%============================================================================

\begin{document}
\setsecnumdepth{subsubsection}
\maketitle
%\setsecnumdepth{subsection}
\thispagestyle{courseinformationtitle}

\section{The Instructor}
\begin{tabular}{rl|rl}
Name: & Dr. Dale J. Brugh & Office Location: & SCSC 262 \\
Email: & djbrugh@owu.edu & Office Phone: & 740-368-3530 \\
\end{tabular}

\section{Laboratory Meetings}
\begin{tabular}{crcrl}
M & 1:10 p.m. & to & 4:00 p.m. & SCSC 355 \\
\end{tabular}

\section{Course Website}
The course web site is at http://dephlo.net/pchemlab. The course website is an important extension of this syllabus and should be read carefully. 

\section{Laboratory Materials}
The following items will be used in this laboratory. 
\begin{enumerate}
\item \textit{Exploring Chemistry with Electronic Structure Methods} by James B. Foresman and \AE leen Frisch, Second Edition, Gaussian, Inc. (1996). ISBN-10: 0963676938, ISBN-13: 978-0963676931.
\item \emph{Physical Chemistry} with MasteringChemistry by Thomas Engel and Phillip Reid, Third Edition, Prentice Hall (2013).  ISBN-13: 9780321766205
\item \emph{Quanta, Matter, and Change} by Peter Atkins, Julio de Paula, and Ronald Friedman, First Edition, W.H. Freeman and Company (2009). ISBN-10: 07167-6117-3, ISBN-13: 978-0-7167-6117-4.
\end{enumerate}

\section{Laboratory Prerequisites}
To take this laboratory course you must have passed one semester of physical chemistry that covers quantum mechanics. The easiest way to meet this prerequisite is to have credit for CHEM350. 

\section{Laboratory Content}
Chemistry 352 is an introduction to molecular modeling and electronic structure methods in chemistry using Gaussian 09. The emphasis is on answering chemical questions using available tools. The topics to be covered are listed in Table~\ref{tab:topics}. These may not be covered in this order, and not all topics will be covered with equal depth. 

\begin{table}[h]
\caption{\sffamily Topics covered in Chemistry 352.}
\label{tab:topics}
\begin{tabular}{l|l} \toprule
Using Gaussian & Basis Sets \\
Specifying Molecular Structure & Selecting Methods \\
Energy Calculations & Chemical Reactivity \\
Geometry Optimization & Excited States \\
Frequency Calculations & Solutions \\ 
\bottomrule
\end{tabular}
\end{table}

\section{Laboratory Goals}
This laboratory has several goals. Each action I take is designed to move us closer to achieving one or all of these goals. My goals in this laboratory are
\begin{enumerate}
\item To provide you with an overview of molecular modeling and electronic structure techniques for computing properties of molecules,
\item To give you practice carrying out molecular modeling and electronic structure calculations on molecules using a research-grade tool,
\item To make you better able to recognize, understand, and interpret molecular modeling and electronic structure results presented in the literature,
\item To make you better able to formulate chemical questions that can be answered with molecular modeling and electronic structure methods, and
\item To provide you with the tools to continue your education in chemistry in the very best graduate programs on planet Earth.
\end{enumerate}

\section{Laboratory Expectations}
You are expected to read all the assigned readings, to work all the assigned exercises, and to come to each laboratory meeting prepared to participate fully. You are expected to display a high level of independence and a strong willingness to solve problems through your own troubleshooting. In short, you are expected to be professional. It is also assumed that you are passionately interested in chemistry.

\section{Weekly Routine}
During laboratory meetings you will work on projects intended to illustrate particular concepts in computational chemistry and to challenge your problem solving abilities. A pre-lab assignment will typically be assigned for each laboratory meeting. A report will be due upon completion of each project. For the last one-third of the semester, laboratory meetings will be dedicated to work on a single independent project of your choosing. There will be no weekly pre-lab assignments or weekly reports for this part of the course. A detailed schedule can be found on the course website.

\section{Things I Grade}
You and I will determine your progress in this laboratory using the scores derived from evaluating your laboratory notebook, reports, final project, and final exam. This section provides details for each of these.

\subsection{Laboratory Notebook}
Each of you will maintain a laboratory notebook as a complete record of your work both in and out of the laboratory. All your work should be recorded in a laboratory notebook. Guidelines for keeping a laboratory notebook can be found on the course web site. I will evaluate your laboratory notebook and score it on a 100 point scale at least 2 times during the semester. Evaluations will take place at random times. Since your laboratory notebook will be kept as a Google document, I can evaluate your notebook at any time without warning.

\subsection{Reports}
For each laboratory project, you must complete a brief written report consisting of three sections: abstract, procedure, and results. Reports will be evaluated on a 100 point scale for clarity, presentation, and accuracy. Details for preparing reports and their due dates and times can be found on the course website.

\subsection{Final Project}
During the last third of the semester, laboratory meetings will be dedicated to work on a self-designed final project. Your work on this final project must be presented in a formal written report and an oral presentation. The combination of your formal report, oral presentation, and supporting work will be evaluated on a 200 point scale with 100 points dedicated to the report and 100 points dedicated to the presentation. 

\subsection{Final Exam}
A final exam will be given during one of the final exam time slots for MWF afternoon courses. The final exam will consist of questions about carrying out molecular modeling and electronic structure calculations. It will also consist of a practical in which you have to carry out computations. You will be free to use your laboratory notebook during the exam; it is therefore in your best interest to keep a detailed laboratory record of your work. 

\section{Things I Do Not Grade}
You will spend a lot of time completing pre-lab assignments, but there is no separate grade for pre-lab assignments. 

\section{Course Letter Grade}
Your course score, from which your course letter grade is derived, will be calculated as a weighted average of the scores you earn on evaluations of your laboratory notebook, reports, final project, and final exam. The weights to be applied to each evaluation item are shown in Table~\ref{tab:weights}.

\begin{table}[h]
\caption{\sffamily Weight of items contributing to your final course grade.}
\label{tab:weights}
\begin{tabular}{r|l} \toprule
\textbf{Evaluation Item} & \textbf{Weight} \\ \hline
Laboratory Notebook & 25\% \\
Weekly Reports & 25\% \\
Final Project & 25\% \\
Final Exam & 25\% \\
\bottomrule
\end{tabular}
\end{table}

At the end of the course your course score is translated into a course letter grade according to the procedure explained at dephlo.net/lettergrades. 

The course letter grade you earn is not a reflection of your worth as a human being. If you work hard throughout the course and learn something about chemistry or yourself, the outcome is positive regardless of the letter grade earned. 

Your effort in this course is not a formal part of determining your course letter grade except as measured through performance. This will frustrate some of you. Keep in mind, however, that effort and attitude do matter if your course score is borderline. See dephlo.net/lettergrades for more detail. 

\end{document}